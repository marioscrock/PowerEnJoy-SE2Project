\section{Introduction}
\subsection{Purpose of this document}

\subsection{Scope}
PowerEnJoy is a car-sharing service that exclusively employs electric cars; we are going to develop a web-based software system that will provide the functionalities normally provided by car-sharing services, such as allowing the user to register to the system in order to access it, showing the cars available near a given location and allowing a user to reserve a car before picking it up.
A screen located inside the car will show in real time the ride amount of money to the user. When the user reaches a predefined safe area and exits the car, the system will stop charging the user and will lock the car. The system will provide information about charging station location where the car can be plugged after the ride and incentivize virtuous behaviours of the users with discounts\todo{citation RASD}
\subsection{Glossary}
\subsubsection{Definitions}

\subsubsection{Acronyms}
	\begin{description}
		\item [RASD:] Requirements Analysis and Specification Document
		\item [DD:] Design Document
		\item [API:] Application Programming Interface
		\item [GPS:] Global Position System
		\item [DB:] DataBase
		\item [DBMS:] DataBase Management System
		\item [GIS:] Geographic Information System
	\end{description}
\subsubsection{Abbreviations}
	\begin{description}
		\item [m:] meters (with multiples and submultiples)
		\item [w.r.t.:] with respect to
		\item [i.d.:] id est
		\item [i.f.f.:] if and only if
		\item [e.g.:] exempli gratia
		\item [etc.:] et cetera
	\end{description}

\subsection{Reference documents}
All domain assumptions, goals, requirements and system components' definition are all described in the \emph{PowerEnJoy: Requirements Analysis and Specification Document} - \mbox{D. Piantella}, \mbox{M. Scrocca}, \mbox{M.R. Vendra}

\subsection{Document overview}
This document is structured as
\begin{enumerate}
	\item \textbf{Introduction}: it provides an overview of the entire document
	\item \textbf{Architectural design}: it describes different views of high level components and their interactions
	\item \textbf{Algorithm design}: it focuses on the definition of the most relevant algorithmic part
	\item \textbf{User interface design}: it provides an overview on how the user interfaces of our system will look like
	\item \textbf{Requirements traceability}: it explains how the requirements we have defined in the RASD map to the design elements that we have defined in this document.
	\item \textbf{Appendices}: it contains references, software and tools used and hours of work per each team member
\end{enumerate}