\section[Project estimations]{Project size, cost and effort estimation}

\subsection{Size estimation: Function Points}
\paragraph{What are FP}Given the assumption that the dimension of software can be characterized based on the functionalities that it offers, Function Points (FP) analysis is a process used to estimate software functional size. The most pervasive version of FP analysis is regulated by the International Function Point User Group (IFPUG).

Functional components can be clustered as: Internal Logical Files (ILF), External Interface Files (EIF), External Inputs (EI), External Outputs (EO) and External Inquiries (EQ).

Each functional component is associated with a \emph{complexity level} based on its associated file and data structure and cardinality. Measures and dimensions are: Data Element Types (DET), File Types Referenced (FTR) and Record Element Types (RET).

Each complexity level is then converted into a \emph{weight coefficient} which is used in the FP formulae to compute overall estimations.

\paragraph{Complexity levels and their weight}\autoref{tbl:complexityILF_EIF}, \autoref{tbl:complexityEI} and \autoref{tbl:complexityEO_EQ} describe how complexity level is mapped on each function component. 

In \autoref{tbl:complexityWeight} is shown how each function component is then assigned a weight
according to its complexity.


\begin{longtable}{cccc}
\toprule
\multicolumn{1}{c}{} & 
\multicolumn{3}{c}{DET}\\
\midrule
RET & 1-19 & 20-50 & 51+ \\
\midrule
1	&	Low	&	Low		&	Avg \\
2-5	&	Low	&	Avg		&	High \\
6+	&	Avg	&	High	&	High\\
\bottomrule
\caption{Complexity for ILF and EIF}
\label{tbl:complexityILF_EIF}
\end{longtable}

\begin{longtable}{cccc}
\toprule
\multicolumn{1}{c}{} & 
\multicolumn{3}{c}{DET}\\
\midrule
FTR & 1-4 & 5-15 & 16+ \\
\midrule
0-1	&	Low	&	Low		&	Avg \\
2	&	Low	&	Avg		&	High \\
3+	&	Avg	&	High	&	High\\
\bottomrule
\caption{Complexity for EI}
\label{tbl:complexityEI}
\end{longtable}

\begin{longtable}{cccc}
\toprule
\multicolumn{1}{c}{} & 
\multicolumn{3}{c}{DET}\\
\midrule
FTR & 1-5 & 6-19 & 20+ \\
\midrule
0-1	&	Low	&	Low		&	Avg \\
2-3	&	Low	&	Avg		&	High \\
4+	&	Avg	&	High	&	High\\
\bottomrule
\caption{Complexity for EO and EQ}
\label{tbl:complexityEO_EQ}
\end{longtable}


%\begin{longtable}{cccc}
%\toprule
%RET & \multicolumn{3}{c}{DET} \\
%\cmidrule(lr){2-4}
%& 1-19 & 20-50 & 51+ \\
%\midrule
%1	&	Low	&	Low		&	Avg \\
%2-5	&	Low	&	Avg		&	High \\
%6+	&	Avg	&	High	&	High\\
%\bottomrule
%\end{longtable}

\begin{longtable}{cccc}
\toprule
Component				&	Low	&	Average	&	High \\
\midrule
External Inputs			&	3	&	4		&	6 \\
External Outputs		&	4	&	5		&	7 \\
External Inquiries		&	3	&	4		&	6 \\
Internal Logic Files	&	7	&	10		&	15 \\
External Interface Files&	5	&	7		&	10 \\
\bottomrule
\caption{Function component complexity weight assignment}
\label{tbl:complexityWeight}
\end{longtable}

\paragraph{Computing FP}The Function Point (FP) is computed as
$$ FP = \sum_{i \in I} \sum_{j \in i} w_{ij}$$
where $ I = \{ILF,EIF,EI,EO,EQ\}$ and $w_{ij}$ is the weight associated with the $j$-th function component of type $i$.


\paragraph{Computing SLOC}To be able to convert FP into Source Lines Of Code (SLOC) a language-dependent factor must be taken into consideration. Since J2EE was chosen as coding language for our system, we report in \autoref{tbl:sloc_fp_j2ee} the industry gearing distribution of J2EE SLOC/FP factor. \cite{QSM}

\begin{longtable}{cccc}
\toprule
\textbf{Average} & \textbf{Median} & \textbf{Low} & \textbf{High}\\
\midrule
46 & 49 & 15 & 67\\
\bottomrule
\caption{J2EE SLOC/FP factor distribution}
\label{tbl:sloc_fp_j2ee}
\end{longtable}

\subsubsection{Internal Logic Files (ILFs)}

\subsubsection{External Interface Files (EIFs)}

\subsubsection{External Inputs (EIs)}

\subsubsection{External Inquiries (EQs)}
 
\subsubsection{External Outputs (EOs)} 

\subsubsection{Overall estimation}

\subsection{Cost and effort estimation: COCOMO II} 

\subsubsection{Scale Drivers}

\subsubsection{Cost Drivers}

\subsubsection{Effort equation}

\subsubsection{Schedule estimation }