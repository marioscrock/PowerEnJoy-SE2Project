\section[Project estimations]{Project size, cost and effort estimation}

This section provides an estimation for the size, cost and effort of the project based upon techniques that support the estimation procedure. The \mbox{\emph{Function Points}} method is used to estimate the size of the project while the  \mbox{\emph{COCOMO II}} method is used for cost and effort estimations..

\subsection{Size estimation: Function Points}
Given the assumption that the dimension of software can be characterized based on the functionalities that it offers, Function Points (FP) analysis is a process used to estimate software functional size. The most pervasive version of FP analysis is regulated by the International Function Point User Group (IFPUG).

Functional components can be clustered as: Internal Logical Files (ILF), External Interface Files (EIF), External Inputs (EI), External Outputs (EO) and External Inquiries (EQ).

Each functional component is associated with a \emph{complexity level} based on its associated file and data structure and cardinality. Measures and dimensions are: Data Element Types (DET), File Types Referenced (FTR) and Record Element Types (RET).

Each complexity level is then converted into a \emph{weight coefficient} which is used in the FP formulae to compute overall estimations.

\paragraph{Complexity levels and their weight}\autoref{tbl:complexityILF_EIF}, \autoref{tbl:complexityEI} and \autoref{tbl:complexityEO_EQ} describe how complexity level is mapped on each function component. 

In \autoref{tbl:complexityWeight} is shown how each function component is then assigned a weight
according to its complexity. \\


\begin{longtable}{cccc}
\toprule
\multicolumn{1}{c}{} & 
\multicolumn{3}{c}{\textbf{DET}}\\
\midrule
\textbf{RET} & 1-19 & 20-50 & 51+ \\
\midrule
1	&	Low	&	Low		&	Avg \\
2-5	&	Low	&	Avg		&	High \\
6+	&	Avg	&	High	&	High\\
\bottomrule \\
\caption{Complexity for ILF and EIF}
\label{tbl:complexityILF_EIF}
\end{longtable}

\begin{longtable}{cccc}
\toprule
\multicolumn{1}{c}{} & 
\multicolumn{3}{c}{\textbf{DET}}\\
\midrule
\textbf{FTR} & 1-4 & 5-15 & 16+ \\
\midrule
0-1	&	Low	&	Low		&	Avg \\
2	&	Low	&	Avg		&	High \\
3+	&	Avg	&	High	&	High\\
\bottomrule \\
\caption{Complexity for EI}
\label{tbl:complexityEI}
\end{longtable}

\clearpage

\begin{longtable}{cccc}
\toprule
\multicolumn{1}{c}{} & 
\multicolumn{3}{c}{\textbf{DET}}\\
\midrule
\textbf{FTR} & 1-5 & 6-19 & 20+ \\
\midrule
0-1	&	Low	&	Low		&	Avg \\
2-3	&	Low	&	Avg		&	High \\
4+	&	Avg	&	High	&	High\\
\bottomrule \\
\caption{Complexity for EO and EQ}
\label{tbl:complexityEO_EQ}
\end{longtable}


%\begin{longtable}{cccc}
%\toprule
%RET & \multicolumn{3}{c}{DET} \\
%\cmidrule(lr){2-4}
%& 1-19 & 20-50 & 51+ \\
%\midrule
%1	&	Low	&	Low		&	Avg \\
%2-5	&	Low	&	Avg		&	High \\
%6+	&	Avg	&	High	&	High\\
%\bottomrule
%\end{longtable}

\begin{longtable}{cccc}
\toprule
Component				&	Low	&	Average	&	High \\
\midrule
External Inputs			&	3	&	4		&	6 \\
External Outputs		&	4	&	5		&	7 \\
External Inquiries		&	3	&	4		&	6 \\
Internal Logic Files	&	7	&	10		&	15 \\
External Interface Files&	5	&	7		&	10 \\
\bottomrule \\
\caption{Function component complexity weight assignment}
\label{tbl:complexityWeight}
\end{longtable}

\paragraph{Computing FP}The Function Point (FP) is computed as
$$ FP = \sum_{i \in I} \sum_{j \in i} w_{ij}$$
where $ I = \{ILF,EIF,EI,EO,EQ\}$ and $w_{ij}$ is the weight associated with the $j$-th function component of type $i$.


\paragraph{Computing SLOC}To be able to convert FP into Source Lines Of Code (SLOC) a language-dependent factor must be taken into consideration. Since J2EE was chosen as coding language for our system, we report in \autoref{tbl:sloc_fp_j2ee} the industry gearing distribution of J2EE SLOC/FP factor. \cite{QSM}

\begin{longtable}{cccc}
\toprule
\textbf{Average} & \textbf{Median} & \textbf{Low} & \textbf{High}\\
\midrule
46 & 49 & 15 & 67\\
\bottomrule \\
\caption{J2EE SLOC/FP factor distribution}
\label{tbl:sloc_fp_j2ee}
\end{longtable}

\paragraph{Notes} The \emph{Function Points} analysis described does not take into account aspects regarding the users interface.
\clearpage

\subsubsection{Internal Logic Files (ILFs)}
\label{sec:ILFs}

The \emph{Function Points} method explains how to identify ILFs function points through the table provided defining ILF, DET and RET \cite{FP}:
\begin{itemize}
	\item An \emph{internal logical file} (ILF) is a user identifiable group of logically related data or control information maintained within the boundary of the application; the primary intent of an ILF is to hold data maintained through one or more elementary processes of the application being counted 
	\item A \emph{data element type} (DET) is a unique user recognizable, non-repeated field
	\item A {record element type} (RET) is a user recognizable subgroup of data elements within an ILF or EIF
\end{itemize}

\paragraph{\emph{Notes}} Given that definitions, ILFs function points must be calculated independently from the technology chosen for the representation of data, omitting fields used by the system to manage data but non user recognizable (e.g. primary key generated to distinguish tuples in a relational database but without a meaning outside the actual representation).\\ 

This section describes the ILFs and related RETs and DETs identified for the \emph{PowerEnJoy system}.

\paragraph{User} The system need to store information about user's personal info (name and surname, date of birth and place of birth, address), user's login info (username and password), user's payment info (credit card number), user's driving license (driving license number), user's status (status and banned reason).

\paragraph{Rent payment} The system needs to store information about three subgroups of data related to rent payments:
\begin{itemize}
	\item \textbf{Payments}: for each payment the system needs to store data about the rent and the user related to the payment, payment status, payment timestamp, base cost of the ride, applied discount and applied fee
	\item \textbf{Fees}: for each fee applicable to a rent the system needs to store data about fixed rate and percentage rate of the fee
	\item \textbf{Discount}: for each discount applicable to a rent the system needs to store data about fixed rate and percentage rate of the fee
\end{itemize}

\paragraph{Rent} For each rent the system needs to store data about start timestamp and start location, end timestamp and end location, payment related to the rent, user performing the rent, car rented and eventually charging station related to the money saving option.

\paragraph{Reservation} For each active reservation the system needs to store data about start timestamp, user who made the reservation, car reserved and eventually charging station related to the money saving option.

\paragraph{Car} The system needs to store information about two subgroups of data related to cars:
\begin{itemize}
	\item \textbf{Car data}: for each car the system needs to store data about the car status, the GPS position, the battery level, the license number, the model and a last-seen-timestamp to know last time information about other fields has been updated
	\item \textbf{Failures}: for each failure (fixed or not) the system needs to store data about problem description, report date and eventually the date when the failure has been fixed
\end{itemize}

\textbf{\emph{Notes}} Some data stored by the system are retrieved from cars trough the provided API primitives, these data represent a cache copy of the actual data of the car which instead are considered as an EIF. 

\paragraph{Charging Station} For each charging station the system needs to store data about GPS location and number of plugs.

\paragraph{Safe Area} The system needs to store information about two subgroups of data related to safe areas:
\begin{itemize}
	\item \textbf{Closed polygonal chain}: for each safe area the system needs to store data about points composing the safe area and their order in the closed polygonal chain composing it
	\item \textbf{Point of the polygonal chain}: for each point the system needs to store the GPS coordinates
\end{itemize}

\begin{longtable}{ccccc}
\toprule
\textbf{ILF}			&	RET	&	DET	&	Complexity  & \textbf{FPs}\\
\midrule
User			&	1	&	10		&	Low & 7 \\
Rent payment & 3 & 11 & Low & 7 \\
Rent & 1 & 8 & Low & 7 \\
Reservation & 1 & 4 & Low & 7 \\
Car & 2 & 9 & Low & 7\\
Charging Station & 1 & 3 & Low & 7 \\
Safe Area & 2 & 3 & Low & 7 \\
\midrule
\textbf{Total} & & & & \textbf{49}\\
\bottomrule \\
\caption{Function points ILFs}
\label{tbl:ilfFP}
\end{longtable}

\clearpage

\subsubsection{External Interface Files (EIFs)}
The \emph{Function Points} method explains how to identify EIFs function points through the table provided defining EIF, DET and RET.

An \emph{external interface file} (EIF) is a user identifiable group of logically related data or control information referenced by the application, but maintained within the boundary of another application. The primary intent of an EIF is to hold data referenced through one or more elementary processes within the boundary of the application counted. This means an EIF counted for an application must be in an ILF in another application \cite{FP}.

DET and RET are defined as described in the \hyperref[sec:ILFs]{ILFs section}.\\

This section describes the EIFs and related RETs and DETs identified for the \emph{PowerEnJoy system}.

\paragraph{Location} The system relies on data about mapping of location and GPS position of the GIS API.

\paragraph{Car} The system relies on data provided by the car API: unique car identifier, GPS position, engine status (on, off), current number of passengers, maximum number of passengers, battery level (percentage), charging status (in charge, not in charge), door locking status (locked, unlocked), door closing status (open, closed).

\paragraph{Notes} The system relies on data about maps provided by the GIS API but only needs a map reference to allow the user client application to show a map centred in a given position, with a given zoom and a set of pointer and areas (data managed by the system and already considered as ILF), therefore map data are not considered as an EIF.

\begin{longtable}{ccccc}
\toprule
\textbf{EIF}			&	RET	&	DET	&	Complexity  & \textbf{FPs}\\
\midrule
Location			&	1	&	2		&	Low & 7 \\
Car & 1 & 9 & Low & 7 \\
\midrule
\textbf{Total} & & & & \textbf{14}\\
\bottomrule \\
\caption{Function points EIFs}
\label{tbl:eifFP}
\end{longtable}

\clearpage

\subsubsection{External Inputs (EIs)}
\label{sec:EIs}

The \emph{Function Points} method explains how to identify EIs function points through the table provided defining EI, FTR and DET \cite{FP}:
\begin{itemize}
	\item An \emph{external input} (EI) is an elementary process that processes data or control information that comes from outside the application boundary. The primary intent of an EI is to maintain one or more ILFs and/or to alter the behavior of the system.
	\item \emph{A file type referenced} (FTR) is:
	\begin{itemize}
 		\item An internal logical file read or maintained by a transactional function \emph{or}
 		\item An external interface file read by a transactional function
	\end{itemize}
	\item A \emph{data element type} (DET) is a unique user recognizable, non-repeated field
\end{itemize} 

This section describes the EIs and related FTRs and DETs identified for the \emph{PowerEnJoy system} grouped by the user category.

\paragraph{Guest user}
\begin{itemize}
	\item \textbf{Register} to the \emph{PowerEnjoy system}: to be processed by the system only requires users data to instantiate a new user record
\end{itemize}

\paragraph{Registered user}
\begin{itemize}
	\item \textbf{Authenticate} to the \emph{PowerEnjoy system}: to be processed by the system only requires users data to check inserted credentials
	\item \textbf{Edit user info}: to be processed by the system only requires users data to modify information stored by the system as requested
	\item \textbf{See available cars}: to be processed by the system it requires cars info (data stored in the system if recently updated, or data retrieved from cars otherwise), charging stations and safe areas data to show it on the map, and eventually data provided by GIS API to decode the position inserted by the user
	\item \textbf{Reserve a car} (we consider the most complete scenario with money saving option enabled): to be processed by the system it requires cars info, access to active reservations data to instantiate a reservation record, charging stations data to provide the recommended destination and eventually data provided by GIS API to decode the position inserted by the user
	\item \textbf{Reservation expired}: to be processed by the system it requires active reservations info to delete the reservation and payments data to instantiate a new payment record
	\item \textbf{Unlock a car}: to be processed by the system it requires active reservations data to check if the request is allowed for the user making it, requires cars info and access to rents data to instantiate a new rent record
	\item \textbf{Start a rent} (the system is notified that the user has ignited the car engine): to be processed by the system it requires rents data to insert start timestamp and location and car data to change car status
	\item \textbf{End a rent} (the system is notified that the user has turned engine off, has leaved the car and has closed doors): to be processed by the system it requires to retrieve data from cars, requires to modify rents data and access to payments data to instantiate a new payment record and calculate the applicable fees and discount and the final total cost
\end{itemize}

\paragraph{Customer care user}
\begin{itemize}
	\item \textbf{Ban or enable a user}: to be processed by the system it only requires users data to change user status
	\item \textbf{Tag a car as \emph{Not available}}: to be processed by the system it only requires cars data to change car status, instantiate a new failures record and its fields
\end{itemize}

\paragraph{Maintenance API}
\begin{itemize}
	\item \textbf{Retrieve \emph{Not Available} cars}: to be processed by the system it requires cars and failures data but also to retrieve from cars software keys to access them (so it can't be considered as a an EQ)
	\item \textbf{Tag a car as \emph{Available}}: to be processed by the system it only requires cars data to change car status and mark the failure as fixed
\end{itemize}

\paragraph{Car Event}
\begin{itemize}
	\item \textbf{Critical battery level}: to be processed by the system it only requires cars data to change car status as \emph{Not available}, instantiate a new failures record and its fields
\end{itemize}

\begin{longtable}{ccccc}
\toprule
\textbf{EI}	& FTR & DET & 	Complexity  & \textbf{FPs}\\
\midrule
Register & 1 & 5-15 &	Low & 3 \\
Edit user info & 1 & 1-4 &	Low & 3 \\
Authenticate &1 & 1-4 & Low & 3 \\
See available cars & 3+ & 5-15 & High & 6 \\
Reserve a car & 3+ & 5-15 & High & 6 \\
Reservation expired & 2 & 5-15 & Avg & 4 \\
Unlock a car & 3+ & 5-15 & High & 6 \\
Start a rent & 2 & 1-4 &	Low & 3 \\
End a rent & 3+ & 5-15 & High & 6 \\
Ban or enable a user & 1 & 1-4 &	Low & 3 \\
Tag a car as \emph{Not available} & 1 & 1-4 &	Low & 3 \\
Retrieve \emph{Not available} cars & 2 & 5-15 & Avg & 4\\
Tag a car as \emph{Available} & 1 & 1-4 &	Low & 3\\
Critical battery level & 1 & 1-4 &	Low & 3\\
\midrule
\textbf{Total} & & & &  \textbf{56}\\
\bottomrule 
\caption{Function points EIs}
\label{tbl:eiFP}
\end{longtable}


\subsubsection{External Outputs (EOs)} 

The \emph{Function Points} method explains how to identify EOs function points through the table provided defining EO, FTR and DET.

An \textit{external output} (EO) is an elementary process that sends data or control information outside the application boundary. The primary intent of an external output is to present information to a user through processing logic other than, or in addition to, the retrieval of data or control information. The processing logic must contain at least one mathematical formula or calculation, create derived data, maintain one or more ILFs or alter the behavior of the system \cite{FP}.

DET and FTR are defined as described in the \hyperref[sec:EIs]{EIs section}.\\

This section describes the EOs and related FTRs and DETs identified for the \emph{PowerEnJoy system} grouped by the user category.

\paragraph{Payment API}
\begin{itemize}
	\item \textbf{Make a payment procedure}: to be executed by the system through the payment API it requires data about the payment to be made and information about user's payment method. It also requires to modify user status if the payment procedure fails 
\end{itemize}

\paragraph{GIS API}
\begin{itemize}
	\item \textbf{Decode an address as GPS position}: to be executed by the system through the GIS API it requires the address to be decoded and access to location EIF related to GIS API
	\item \textbf{Obtain a map reference}: to be executed by the system and to retrieve the reference through the GIS API it requires the GPS position of the user to center the map, the zoom and information about cars (data stored in the system if recently updated, or data retrieved from cars otherwise), safe areas and charging stations
\end{itemize}

\paragraph{Cars}
\begin{itemize}
	\item \textbf{Car primitive}: to be executed by the system through the cars provided API it requires access to data about cars to identify car to communicate with and eventually to update fields related. Number of DET related may depend on the primitive called so we consider that EO of \emph{Average} difficulty
\end{itemize}

\clearpage 

\begin{longtable}{ccccc}
\toprule
\textbf{EO}	& FTR & DET & 	Complexity  & \textbf{FPs}\\
\midrule
Make a payment procedure & 2-3 & 6-19 & Avg & 5\\
Decode an address as GPS position & 1 & 1-5 & Low & 4 \\
Obtain a map reference & 4+ & 6-19 & Avg & 7\\
Car primitive & 1 & - & Avg & 5  \\
\midrule
\textbf{Total} & & & &  \textbf{21}\\
\bottomrule \\
\caption{Function points EOs}
\label{tbl:eoFP}
\end{longtable}


\subsubsection{External Inquiries (EQs)}
The \emph{Function Points} method explains how to identify EQs function points through the table provided defining EQ, FTR and DET.

An \textit{external inquiry} (EQ) is an elementary process that sends data or control information outside the application boundary. The primary intent of an external inquiry is to present information to a user through the retrieval of data or control information from an ILF of EIF. The processing logic contains no mathematical formulas or calculations, and creates no derived data. No ILF is maintained during the processing, nor is the behavior of the system
altered \cite{FP}.

DET and FTR are defined as described in the \hyperref[sec:EIs]{EIs section}.\\

This section describes the EQs and related FTRs and DETs identified for the \emph{PowerEnJoy system} grouped by the user category.

\paragraph{Registered user}
\begin{itemize}
	\item \textbf{See personal information}: to be processed by the system it requires the userID and data about users
	\item \textbf{See rent history}: to be processed by the system it requires the userID and data about rents
	\item \textbf{See payment history}: to be processed by the system it requires the userID and data about payments
\end{itemize}

\paragraph{Customer care user}
\begin{itemize}
	\item \textbf{See all user IDs}:  to be processed by the system it requires data about users
	\item \textbf{See information about a user}: to be processed by the system it requires the userID and data about users
	\item \textbf{See rent history of a user}: to be processed by the system it requires the userID and data about rents
	\item \textbf{See payment history of a user}: to be processed by the system it requires the userID and data about payments
\end{itemize}

\paragraph{Notes} Similiar EQs between users and customer care operator are considered separately because are requested from different server software components and they also differ in information retrieved as EQ result. \\

\begin{longtable}{ccccc}
\toprule
\textbf{EQ}	& FTR & DET & 	Complexity  & \textbf{FPs}\\
\midrule
See own information (\emph{user}) & 1 & 6-19 & Low & 4\\
See own rent history (\emph{user}) & 1 & 6-19 & Low & 4\\
See own payment history (\emph{user}) & 1 & 6-19 & Low & 4\\
See all user IDs & 1 & 1-5 & Low & 4\\
See information about a user & 1 & 6-19 & Low & 4\\
See rent history of a user & 1 & 6-19 & Low & 4\\
See payment history of a user & 1 & 6-19 & Low & 4\\
\midrule
\textbf{Total} & & & &  \textbf{28}\\
\bottomrule \\
\caption{Function points EQs}
\label{tbl:eqFP}
\end{longtable}

\clearpage 

\subsubsection{Overall estimation}

The following table summarizes FPs count identified through the analysis for each component.\\
\begin{longtable}{cc}
\toprule
\textbf{Function Type Value}  & \textbf{FPs}\\
\midrule
Internal Logic Files & 49 \\
External Interface Files & 14 \\
External Inputs & 56 \\
External Outputs &  21\\
External Inquiries & 28 \\
\midrule
\textbf{Total} & \textbf{168}\\
\bottomrule \\
\caption{Function points}
\label{tbl:FP}
\end{longtable}

Through the \emph{SLOC} estimation formula we can therefore estimate an average and an upper bound for the size of the project. 

 $$SLOC = (\sum_{i \in I} \sum_{j \in i} w_{ij}) \cdot K$$
where 
\begin{itemize}
	\item I = \{ILF,EIF,EI,EO,EQ\}
	\item $w_{ij}$ is the weight associated with the $j$-th function component of type $i$
	\item $K$ is the industry gearing distribution of J2EE $\frac{SLOC}{FP}$ factor
\end{itemize} 

 \begin{align*}
 SLOC_{avg} &= (\sum_{i \in I} \sum_{j \in i} w_{ij}) \cdot K = 168 \cdot 46 = 7728 \\ 
  SLOC_{max} &= (\sum_{i \in I} \sum_{j \in i} w_{ij}) \cdot K = 168 \cdot 67 =11256
  \end{align*}
