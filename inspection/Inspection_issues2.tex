\subsection*{[2.9] Class and Interface Declarations}
\subsubsection*{[27] Check that the code is free of duplicates, long methods, big classes, breaking encapsulation, as well as if coupling and cohesion are adequate.}
\begin{description}
	\item[L 194:] the {\tt processSale} method (considering also further implementations described by \emph{todo}s) is too long and should be divided into different methods in order to decouple logic and improve readability
\end{description}

\subsection*{[2.10] Initialization and Declarations}
\subsubsection*{[31] Check that all object references are initialized before use}
\begin{description}
	\item[L 67:] the {\tt session} object passed as parameter in the constructor is referenced and used several times in the class but it has never checked not to be {\tt null}. This might cause a {\tt NullPointerException}.
\end{description} 

\subsubsection*{[33] Declarations appear at the beginning of blocks. The exception is a variable can be declared in a for loop.}
If a block starts with a debug log operation, we consider the first instruction the next one. Variables not declared at the beginning of a block are the following:
\begin{description}
	\item[L 70, L 71:] in the constructor 
	\item[L 207, L 214, L 224, L 237:] in the {\tt processSale} method 
	\item[L 272, L 279:] in the {\tt getStoreOrgAddress} method 
	\item[L 126:] in the {\tt isOpen} method 
	\item[L 324:] in the {\tt checkPaymentMethodType} method  
	\item[L 421, L 431:] in the {\tt makeCreditCardVo} method 
\end{description}

\subsection*{[2.14] Output Format}
\subsubsection*{[42] Check that error messages are comprehensive and provide guidance as to how to correct the problem.}
\begin{description}
	\item[L 396:] in the {\tt processExternalPayment} method no error message is returned, marked as \emph{todo}.
\end{description}

\subsection*{[2.15] Computation, Comparisons and Assignments}
\subsubsection*{[44] Check that the implementation avoids "brutish programming".}
\begin{description}
	\item[L 53, L 54, L 55:] it is better to define an enumeration than use three {\tt public static final int} variables to define the type of payment
\end{description}

\subsection*{[2.16] Exceptions}
\subsubsection*{[53] Check that the appropriate action are taken for each catch block.}
All {\tt catch} blocks only log the exception to the debugger. Moreover
\begin{description}
	\item[L 390:] in the {\tt processNoPayment} method the {\tt catch (GeneralException e)} block is empty
	\item[L 407:] in the {\tt processExternalPayment} method the \mbox{{\tt catch (GeneralException e)}} block is empty
\end{description}