\section{Introduction}
\subsection{Purpose of this document}
In this document we are going to describe software design and architecture of the PowerEnJoy system.\\
The software architecture of a system is the structure or structures of the system, which comprise software elements, the externally visible properties of those elements, and the relationships among them.\cite{SoftwareArch}

\subsection{Scope}
PowerEnJoy is a car-sharing service that exclusively employs electric cars; we are going to develop a web-based software system that will provide the functionalities normally provided by car-sharing services, such as allowing the user to register to the system in order to access it, showing the cars available near a given location and allowing a user to reserve a car before picking it up.
A screen located inside the car will show in real time the ride amount of money to the user. When the user reaches a predefined safe area and exits the car, the system will stop charging the user and will lock the car. The system will provide information about charging station location where the car can be plugged after the ride and incentivize virtuous behaviours of the users with discounts\cite{RASD}
\subsection{Glossary}
The \emph{PowerEnJoy: Requirements Analysis and Specification Document}\cite{RASD} should be referenced for terms not defined in this section.
\subsubsection{Definitions}
	\begin{description}
		\item [Base cost:] cost before any discount or fee application, obtained from rent time and time-based cost
	\end{description}
\subsubsection{Acronyms}
	\begin{description}
		\item [RASD:] Requirements Analysis and Specification Document
		\item [DD:] Design Document
		\item [API:] Application Programming Interface
		\item [GPS:] Global Position System
		\item [DB:] DataBase
		\item [DBMS:] DataBase Management System
		\item [GIS:] Geographic Information System
		\item [ER:] Entity Relationship Model
		\item [XML:] eXtensible Markup Language
		\item [REST API:] REpresentational State Transfer API
		\item [JAX-RS:] JAVA API for REST Web Services
		\item [ISP:] Internet Service Provider
		\item [ARP:] Address Resolution Protocol
	\end{description}
\subsubsection{Abbreviations}
	\begin{description}
		\item [m:] meters (with multiples and submultiples)
		\item [w.r.t.:] with respect to
		\item [i.d.:] id est
		\item [i.f.f.:] if and only if
		\item [e.g.:] exempli gratia
		\item [etc.:] et cetera
	\end{description}

\subsection{Reference documents}
Context, domain assumptions, goals, requirements and system interfaces are all described in the \emph{PowerEnJoy: Requirements Analysis and Specification Document}.\cite{RASD}\\
Others references are:
\begin{itemize}
	\item IEEE Std 1016-2009 Standard for Information Technology, Systems Design, Software Design Descriptions
\end{itemize}

\subsection{Document overview}
This document is structured as
\begin{enumerate}
	\item \textbf{Introduction}: it provides an overview of the entire document
	\item \textbf{Architectural design}: it describes different views of components and their interactions
	\item \textbf{Algorithm design}: it focuses on the definition of the most relevant algorithmic part
	\item \textbf{User interface design}: it provides an overview on how the user interfaces of our system will look like
	\item \textbf{Requirements traceability}: it explains how the requirements we have defined in the RASD map to the design elements that we have defined in this document.
	\item \textbf{Appendices}: it contains references, software and tools used and hours of work per each team member
\end{enumerate}