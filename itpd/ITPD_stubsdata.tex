\section{Program Stubs and Test Data Required}

Based on the testing strategy and test design, identify any program stubs or special test data required for each integration step.

\subsection{Drivers Required}

\subsection{Data Required}
As specified before in this document, the DataProvider component must be implemented and unit-tested before \todo{giusto?} any integration test can start, furthermore referential integrity is always maintained because \emph{InnoDB} is used as DB engine with the usage of foreign keys constrains, therefore we can consider the data retrieved from the DataProvider component always reliable and well-formed.

The database used during the tests must have the same tables and structure as the production one.

Well formed XML files with the location of charging stations (at least 10) and safe areas (at least 2) are also required.

In order to cover the wide variety of possible conditions, the data inserted in the test DB must include \emph{at least}:
\begin{itemize}
	\item 40 registered users, whereof
	\begin{itemize}
		\item 5 banned users
		\item 30 not banned users
		\item 5 users with email confirmation pending
	\end{itemize}
	\item 35 cars, whereof
	\begin{itemize}
		\item 5 reserved 
		\item 5 not available 
		\item 10 in use 
		\item 15 available, whereof
		\begin{itemize}
			\item 2 with lastSeenTime = 2 hours
			\item 5 with lastSeenTime < 2 hours
			\item 8 with lastSeenTime > 2 hours
		\end{itemize}
	\end{itemize}
	\item 10 failures, whereof
	\begin{itemize}
		\item 5 pending (whereof 2 of the same car)
		\item 5 fixed (whereof 2 of the same car)
	\end{itemize}
	\item 5 active reservation, whereof
	\begin{itemize}
		\item 2 with the money saving option
		\item 3 without the money saving option
	\end{itemize}
	\item 40 rents, whereof
	\begin{itemize}
		\item 10 active rents
		\item 30 concluded rents (whereof 5 of the same car, 3 of the same user, 5 of banned users)
	\end{itemize}
	\item 5 fees
	\item 5 discounts
	\item 35 payments (whereof 5 of the same user), whereof
	\begin{itemize}
		\item 2 with status = pending
		\item 5 with status = rejected
		\item 10 without the money saving option
		\item 10 with the money saving option
		\item 5 related to expired reservations
		\item 30 related to concluded rents, whereof
		\begin{itemize}
			\item 5 without fees or discounts related
			\item 5 with 1 fee and 0 discounts related
			\item 5 with 0 fees and 1 discount related
			\item 5 with 3 fees and 0 discount related
			\item 5 with 0 fees and 3 discount related
			\item 5 with more than 3 fees and more than 3 discounts related
		\end{itemize}
	\end{itemize}
\end{itemize}