\section{List of Issues}
This section reports a list of the issues found in the \emph{WebPosTransaction} class, based on the \emph{Code Inspection Assignment} checklist \cite{Assignment}.
\subsection*{[2.1] Naming Conventions}
\subsubsection*{[1] All class names, interface names, method names, class variables, method variables, and constants used should have meaningful names and do what the name suggests}
\begin{description}
	\item[L 57:] The name of the {\tt CheckoutHelper} type variable {\tt ch} is not meaningful and difficult to understand out of the declaration context
	\item[L 272:] The name of the {\tt List<GenericValue>} type variable {\tt fcp} is not meaningful and difficult to understand out of the declaration context
	\item[L 421:] The name of the {\tt Map<String, Object>} type variable {\tt svcCtx} is not meaningful and difficult to understand out of the declaration context
	\item[L 431:] The name of the {\tt Map<String, Object>} type variable {\tt svcRes} is not meaningful and difficult to understand out of the declaration context
\end{description}

\subsubsection*{[2] If one-character variables are used, they are used only for temporary “throwaway” variables}
\begin{description}
	\item[L 329:] The name chosen for the {\tt Iterator<GenericValue>} returned from {values.iterator()} is {\tt i}; although since this variable is only used inside a loop, it is not a problem
	\item[L 331, 333:] The name chosen for the {\tt GenericValue} returned from {\tt i.next()} is {\tt v}; although since this variable is only used inside a loop, it is not a problem
\end{description}

\subsubsection*{[5] Method names should be verbs, with the first letter of each addition word capitalized}
\begin{description}
	\item[L 162:] The {\tt paidInOut(String type)} method name is not a verb which refers to the actions made inside it. \\Since it seems like it only stores the payment referred to a transaction into the log, it could be renamed {\tt logPayment(String type)}
\end{description}

\subsubsection*{[7] Constants are declared using all uppercase with words separated by an underscore}
\begin{description}
	\item[L 51:] The {\tt resource} constant is not uppercase
	\item[L 52:] The {\tt module} constant is not uppercase
	\item[L 324:] The {\tt externalCode} constant is not uppercase
\end{description}

\subsection*{[2.4] File Organization}

\subsubsection*{[12] Blank lines and optional comments are used to separate sections}
\begin{description}
	\item[L 18,19:] There is a missing space between the {\tt License} section and the {\tt package} statements
\end{description}

\subsubsection*{[13] Where practical, line length does not exceed 80 characters}\label{subsec:lineLen}
\begin{description}
	\item[L 139:] The line is 169 characters long, three high-level breaks should be used, after {\tt "PosTerminalState",} , after {\tt "posTerminalId",} and after {\tt getTransactionId())),}
	\item[L 143:] The line is 104 characters long, even though it is less than 120 characters long, a high-level brake can be applied after {\tt UtilDateTime.nowTimestamp(),}, so it should be broken
	\item[L 199:] The line is 115 characters long, even though it is less than 120 characters long, a high-level brake can be applied after {\tt resource,} , so it should be broken
	\item[L 214:] The line is 105 characters long, even though it is less than 120 characters long, a high-level brake can be applied after {\tt =} , so it should be broken
	\item[L 226:] The line is 183 characters long, two high-level breaks should be used, after {\tt =} and after {\tt getProductStoreId(),}
	\item[L 264:] The line is 128 characters long, one high-level break should be used after {\tt "Facility",} 
	\item[L 274:] The line is 135 characters long, one high-level break should be used after {\tt "FacilityContactMechPurpose",} 
	\item[L 316:] The line is 153 characters long, one high-level break should be used after {\tt "paymentMethodTypeId",}
	\item[L 319:] The line is 140 characters long, one high-level break should be used after {\tt "ProductStorePaymentSetting",}
	\item[L 362:] The line is 92 characters long, even though it is less than 120 characters long, a high-level brake can be applied after {\tt String refNum,} , so it should be broken
	\item[L 374:] The line is 92 characters long, even though it is less than 120 characters long, a high-level brake can be applied after {\tt +} , so it should be broken
	\item[L 386:] The line is 85 characters long, even though it is less than 120 characters long, a high-level brake can be applied after {\tt "] Amount : " +} , so it should be broken
	\item[L 395:] The line is 111 characters long, even though it is less than 120 characters long, a high-level brake can be applied after {\tt String paymentMethodTypeId,} , so it should be broken
	\item[L 403:] The line is 85 characters long, even though it is less than 120 characters long, a high-level brake can be applied after {\tt "] Amount : " +} , so it should be broken
	\item[L 412:] The line is 112 characters long, even though it is less than 120 characters long, a high-level brake can be applied after {\tt String expDate,} , so it should be broken
	\item[L 447:] The line is 116 characters long, even though it is less than 120 characters long, a high-level brake can be applied after {\tt paymentIndex +} , so it should be broken
	\item[L 454:] The line is 90 characters long, even though it is less than 120 characters long, a high-level brake can be applied after {\tt String refNum,} , so it should be broken
	\item[L 463:] The line is 86 characters long, even though it is less than 120 characters long, a high-level brake can be applied after {\tt String refNum,} , so it should be broken
	\item[L 472:] The line is 86 characters long, even though it is less than 120 characters long, a high-level brake can be applied after {\tt String refNum,} , so it should be broken
	\item[L 506:] The line is 83 characters long, even though it is less than 120 characters long, a high-level brake can be applied after {\tt code,} , so it should be broken
	
\end{description}
Note that the comma should be the last character before each break.

\subsubsection*{[14] When line length must exceed 80 characters, it does NOT exceed 120 characters}
\begin{description}
	\item[L 78:] The line is 86 characters long, since it is less than 120 characters long and there is no practical way to break it, in can be left as it is
	\item[L 207:] The line is 86 characters long, since it is less than 120 characters long and there is no practical way to break it, in can be left as it is
\end{description}

\subsection*{[2.5] Wrapping Lines}
\subsubsection*{[16] Higher-level breaks are used}
\begin{description}
	\item[L 282:] It is the only case in which line break is used. As stated \hyperref[subsec:lineLen]{before}, in many other cases these are not used while they should
\end{description}

\subsection*{[2.6] Comments}
\subsubsection*{[18] Comments are used to adequately explain what the class, interface, methods, and blocks of code are doing}
\begin{description}
	\item[Entire class:] There are no comments explaining what the class does
	\item[L 453, 462, 471:] There are unclear comments explaining the method beneath them
	\item[L 81, 202, 205, 212, 222, 236, 245, 261, 332, 365, 388, 416:] There are superficial comments explaining the blocks of code beneath them 
\end{description}

\subsubsection*{[19] Commented out code contains a reason for being commented out and a date it can be removed from the source file if determined it is no longer needed.}
\begin{description}
	\item[L 332:] There is a commented line of code, with no date, nor explanation about the reasons it is commented
\end{description}

\subsection*{[2.7] Java Source Files}

\subsubsection*{[20] Each Java source file contains a single public class or interface}
\begin{description}
	\item[Entire class:] The {\tt WebPosTransaction} class is the only one in the WebPosTransaction.java source file, but we noticed that it imports classes contained in the same source file, specifically both the {\tt ShoppingCart} and the {\tt CartPaymentInfo} classes are located in the ShoppingCart.java source file
\end{description}

\subsubsection*{[23] Check that the javadoc is complete}
\begin{description}
	\item[Entire class:] There is no javadoc in the entire class
\end{description}

\subsection*{[2.9] Class and Interface Declarations}
\subsubsection*{[27] Check that the code is free of duplicates, long methods, big classes, breaking encapsulation, as well as if coupling and cohesion are adequate.}
\begin{description}
	\item[L 194:] the {\tt processSale} method (considering also further implementations described by \emph{todo}s) is too long and should be divided into different methods in order to decouple logic and improve readability
\end{description}

\subsection*{[2.10] Initialization and Declarations}
\subsubsection*{[31] Check that all object references are initialized before use}
\begin{description}
	\item[L 67:] the {\tt session} object passed as parameter in the constructor is referenced and used several times in the class but it has never checked not to be {\tt null}. This might cause a {\tt NullPointerException}.
\end{description} 

\subsubsection*{[33] Declarations appear at the beginning of blocks. The exception is a variable can be declared in a for loop.}
If a block starts with a debug log operation, we consider the first instruction the next one. Variables not declared at the beginning of a block are the following:
\begin{description}
	\item[L 70, L 71:] in the constructor 
	\item[L 207, L 214, L 224, L 237:] in the {\tt processSale} method 
	\item[L 272, L 279:] in the {\tt getStoreOrgAddress} method 
	\item[L 126:] in the {\tt isOpen} method 
	\item[L 324:] in the {\tt checkPaymentMethodType} method  
	\item[L 421, L 431:] in the {\tt makeCreditCardVo} method 
\end{description}

\subsection*{[2.14] Output Format}
\subsubsection*{[42] Check that error messages are comprehensive and provide guidance as to how to correct the problem.}
\begin{description}
	\item[L 396:] in the {\tt processExternalPayment} method no error message is returned, marked as \emph{todo}.
\end{description}

\subsection*{[2.15] Computation, Comparisons and Assignments}
\subsubsection*{[44] Check that the implementation avoids "brutish programming".}
\begin{description}
	\item[L 53, L 54, L 55:] it is better to define an enumeration than use three {\tt public static final int} variables to define the type of payment
\end{description}

\subsection*{[2.16] Exceptions}
\subsubsection*{[53] Check that the appropriate action are taken for each catch block.}
All {\tt catch} blocks only log the exception to the debugger. Moreover
\begin{description}
	\item[L 390:] in the {\tt processNoPayment} method the {\tt catch (GeneralException e)} block is empty
	\item[L 407:] in the {\tt processExternalPayment} method the \mbox{{\tt catch (GeneralException e)}} block is empty
\end{description}