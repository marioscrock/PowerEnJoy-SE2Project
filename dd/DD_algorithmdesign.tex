\section{Algorithm design}
\subsection{Payments}\label{sec:paymentAlgorithms}
\subsubsection{Discount assignment}
Let $d_i=[d_{iFixed},d_{iPercentage}]$ be a general discount with its fixed and percentage components.\\
Let $c$ be the base cost of the rent.\\
Let $D = \lbrace d_1,d_2,\ldots,d_n\rbrace$ be the set of all applicable discounts for a specific rent.\\
Then the final applied discount is $$d = \max_{d_i\in D} \lbrace d_{iFixed}+d_{iPercentage}\cdot c \rbrace$$

\subsubsection{Fee assignment}
Let $f_i=[f_{iFixed},f_{iPercentage}]$ be a general fee with its fixed and percentage components.\\
Let $c$ be the base cost of the rent.\\
Let $F = \lbrace f_1,f_2,\ldots,f_n\rbrace$ be the set of all applicable fees for a specific rent.\\
Then the final applied fee is $$f = \sum_{f_i\in F} ( f_{iFixed} + f_{iPercentage} \cdot c )$$

\subsection{Money saving option}\label{sec:msoAlgorithm}
The Money Saving Option algorithm returns a charging station with available plugs near the destination inserted by the user to ensure a better distribution of cars.

Each charging station is assigned a score considering the distance from the desired destination, the number of cars nearby and its plugs availability. 

The charging station with the highest score is returned.

\lstinputlisting[language=pseudocode]{pseudocode/MSO-pseudocode.txt}
