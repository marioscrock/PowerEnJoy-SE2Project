\subsection{Cost and effort estimation: COCOMO II} 
COnstructive COst MOdel II (COCOMO II) is a model that allows one to estimate the cost, effort, and schedule when planning a new software development activity. COCOMO II is the latest major extension to the original \mbox{COCOMO 81} model published in 1981.

It consists of three submodels, each one offering increased fidelity the further along one is in the project planning and design process \cite{COCOMO}.

Since the system we are to develop is brand new and there is no previous system we have to adopt the \emph{Early design} submodel.

\paragraph{COCOMO effort equation}\label{par:cocomoEquation}This formula estimates the project effort in Person-Months (PM)
$$PM = A \cdot Size^{E} \cdot \prod_{i \in CostDrivers}EM_{i}$$
where 
\begin{itemize}
	\item $A=2.94$ approximates a productivity constant in PM/KSLOC (Person-Months/Kilo-Source Lines of Code)
	\item $Size$ is the estimated size of the project in KSLOC, it can be deducted from Function Points analysis
	\item $E$ is an aggregation of five Scale Factors which is computed as $$E = 0.91 + 0.01 \cdot \sum_{j=1}^{5}{SF_j}$$
	\item $EM$ is the Effort Multiplier derived from each Cost Driver
\end{itemize}

\clearpage

\subsubsection{Scale Factors}
Here are described the five Scale Factors of the COCOMO II model, in \autoref{tbl:cocomoSF} are specified their numerical value according to the model definition.
 
{\scriptsize
\begin{longtable}{c|cccccc}
\toprule
\specialcell{Scale\\Factors}&\specialcell{Very\\Low}&Low&Nominal&High&\specialcell{Very\\High}&\specialcell{Extra\\High}\\
\midrule
PREC	&
\specialcell{thoroughly\\unprecedented\\\textbf{6.20}} & 
\specialcell{largely\\unprecedented\\\textbf{4.96}} & 
\specialcell{somewhat\\unprecedented\\\textbf{3.72}} & 
\specialcell{generally\\familiar\\\textbf{2.48}} & 
\specialcell{largely\\familiar\\\textbf{1.24}} & 
\specialcell{thoroughly\\familiar\\\textbf{0.00}} \\
\midrule
FLEX &
\specialcell{rigorous\\\textbf{5.07}} & 
\specialcell{occasional\\relaxation\\\textbf{4.05}} & 
\specialcell{some\\relaxation\\\textbf{3.04}} & 
\specialcell{general\\conformity\\\textbf{2.03}} & 
\specialcell{some\\conformity\\\textbf{1.01}} & 
\specialcell{general\\goals\\\textbf{0.00}} \\
\midrule
RESL &
\specialcell{little\\(20\%)\\\textbf{7.07}} & 
\specialcell{some\\(40\%)\\\textbf{5.65}} & 
\specialcell{often\\(60\%)\\\textbf{4.24}} & 
\specialcell{generally\\(75\%)\\\textbf{2.83}} & 
\specialcell{mostly\\(90\%)\\\textbf{1.14}} & 
\specialcell{full\\(100\%)\\\textbf{0.00}} \\
\midrule
TEAM &
\specialcell{very\\difficult\\interactions\\\textbf{5.48}} & 
\specialcell{some\\difficult\\interactions\\\textbf{4.38}} & 
\specialcell{basically\\cooperative\\interactions\\\textbf{3.29}} & 
\specialcell{largely\\cooperative\\\textbf{2.19}} & 
\specialcell{highly\\cooperative\\\textbf{1.10}} & 
\specialcell{seamless\\interactions\\\textbf{0.00}} \\
\midrule
PMAT &
\specialcell{SW-CMM\\Level 1\\Lower\\\textbf{7.80}} & 
\specialcell{SW-CMM\\Level 1\\Upper\\\textbf{6.24}} & 
\specialcell{SW-CMM\\Level 2\\\textbf{4.68}} & 
\specialcell{SW-CMM\\Level 3\\\textbf{3.12}} & 
\specialcell{SW-CMM\\Level 4\\\textbf{1.56}} & 
\specialcell{SW-CMM\\Level 5\\\textbf{0.00}} \\
\bottomrule
\caption{\\\label{tbl:cocomoSF}\\COCOMO II scale factors values}
\end{longtable}
}

\paragraph{Precedentedness - PREC} It is high if a product is similar to several projects previously developed by the team. Since we do not have ever develop such a big project, although we have largely understood the product objectives and there is a minimal need of innovative algorithms or innovative data processing architectures, we decide that
\DadoCenter{PREC is \textbf{LOW}}

\paragraph{Development Flexibility - FLEX} It is high if there are no specific constraints to conform to pre-established requirements and external interface specifications. The italian law establishes specific constraints regarding the characteristics and usage of car sharing systems, driving licenses and privacy policy. Requirements specified by the client do not excessively restrict the development process, some external APIs are used in order to fulfill the requirements.\\We decide that
\DadoCenter{FLEX is \textbf{NOMINAL}}

\paragraph{Risk Resolution - RESL} It is high if the project has a good risk management plan, clear definition of budget and schedule, focus on architectural definition. The \hyperref[sec:riskManagement]{Risk Management Plan} identifies generally all critical risk items and determines actions in order to resolve them. As specified in the \hyperref[sec:schedule]{Schedule section} a relevant portion of development process is devoted to establishing architecture, given general product objectives.\\Therefore we decide that
\DadoCenter{REL is \textbf{HIGH}}

\paragraph{Team Cohesion - TEAM} It is high if all project development team members are able to work in a team and share the same vision and commitment. As objectives, cultures, ages and backgrounds are the same for all team members we decide that
\DadoCenter{TEAM is \textbf{VERY HIGH}}

\paragraph{Project Maturity - PMAT} It reflects the CMMI index of the project. Since this project can be considered a managed process, planned and executed according to policies with the usage of adequate and planned resources, stakeholders constantly involved in incremental product reviews, we decide that the CMMI index is Level 2: Managed at the Project Level, therefore
\DadoCenter{PMAT is \textbf{NOMINAL}}

\paragraph{E parameter}In the \autoref{tbl:SFChosenValues} are reported the values chosen for the five Scale Factors
\begin{longtable}{ccc}
\toprule
Scale Factor&Value&Numerical Value\\
\midrule
PREC&Low&4.96\\
FLEX&Nominal&3.04\\
RESL&High&2.83\\
TEAM&Very High&1.10\\
PMAT&Nominal&4.68\\
\midrule
\multicolumn{2}{c}{\textbf{Total}}&\textbf{16.61}\\
\bottomrule\\
\caption{\label{tbl:SFChosenValues}Scale factors chosen values}
\end{longtable}
The $E$ parameter of \hyperref[par:cocomoEquation]{COCOMO effort equation} is computed as
$$E = 0.91 + 0.01 \cdot \sum_{j=1}^{5}{SF_j} = 0.91+0.01\cdot 16.61 = 1.0761$$


\subsubsection{Cost Drivers}
Since we refer to the \emph{Early Design} model of COCOMO II, the cost drivers are obtained averaging their \emph{Post-Architecture} counterparts as shown in \autoref{tbl:costDriverConversion}.
\begin{longtable}{cc}
\toprule
\specialcell{Early Design\\Cost Drivers} & \specialcell{Counterpart Combined Post-Architecture\\Cost Drivers}\\
\midrule
PERS&ACAP, PCAP, PCON\\
RCPX&RELY, DATA, CPLX, DOCU\\
RUSE&RUSE\\
PDIF&TIME ,STOR, PVOL\\
PREX&APEX, PLEX, LTEX\\
FCIL&TOOL, SITE\\
SCED&SCED\\
\bottomrule\\
\caption{\label{tbl:costDriverConversion}Cost drivers conversion}
\end{longtable}

\begin{longtable}{c|ccccccc}
\toprule
\specialcell{Cost\\Driver}&\specialcell{Extra\\Low}&\specialcell{Very\\Low}&Low&Nominal&High&\specialcell{Very\\High}&\specialcell{Extra\\High}\\
\midrule
PERS &2.12 &1.62 &1.26 &1.00 &0.83 &0.63 &0.50\\
RCPX &0.49 &0.60 &0.83 &1.00 &1.33 &1.91 &2.72\\
RUSE &-	   &-	 &0.95 &1.00 &1.07 &1.15 &1.24\\
PDIF &-	   &-    &0.87 &1.00 &1.29 &1.81 &2.61\\
PREX &1.59 &1.33 &1.22 &1.00 &0.87 &0.74 &0.62\\
FCIL &1.43 &1.30 &1.10 &1.00 &0.87 &0.73 &0.62\\
SCED &-	   &1.43 &1.14 &1.00 &1.00 &1.00 &-\\
\bottomrule
\caption{\label{tbl:costDriverConversionValue}COCOMO II early design cost driver values}
\end{longtable}

\paragraph{Personnel Capability - PERS}
\begin{itemize}
	\item \textbf{Analyst Capability - ACAP}: Analysts are personnel who work on requirements, high-level design and detailed design. Since we consider ourselves as good analysts, with the ability to communicate and cooperate, we consider this parameter as \textbf{HIGH}.
	\item \textbf{Programmer Capability - PCAP}: This parameter reflect the capability of the programmers as a team rather than as individuals. Since all team members have already done with profit several team projects before, we consider this parameter as \textbf{HIGH}.
	\item \textbf{Personnel Continuity - PCON}: The rating scale is in terms of the project’s annual personnel turnover. Since the team is composed of 3 members and the average turnover is 2 years, this parameter is \textbf{VERY LOW}.
\end{itemize}

\begin{longtable}{ccc}
\multicolumn{3}{c}{\textbf{PERS}}\\
\toprule
Cost Driver&Value&Numerical Value\\
\midrule
ACAP&High&4\\
PCAP&High&4\\
PCON&Very Low&1\\
\midrule
\multicolumn{2}{c}{\textbf{Total}}&\textbf{9}\\
\bottomrule
\end{longtable}

Therefore PERS is \textbf{NOMINAL}.

\paragraph{Product Reliability and Complexity - RCPX}
\begin{itemize}
	\item \textbf{Required Software Reliability - RELY} This is the measure of the extent to which the software must perform its intended function over a period of time. If the effect of a software failure is only slight inconvenience then RELY is very low. Since a product failure will cause high financial loss to our client's core business, this parameter is \textbf{HIGH}.
	\item \textbf{Data Base Size - DATA} This cost driver attempts to capture the effect large test data requirements have on product development. \todo{the ratio of bytes in the testing database to SLOC in the program.}
	\item \textbf{Product Complexity - CPLX} Complexity is divided into five areas: control operations, computational operations, device-dependent operations, data management operations, and user interface management operations.
	\begin{itemize}
		\item User interface management: simple use of widget set. It is NOMINAL.
		\item Data management operations: large operational database with many updates, possibly with triggers, multi-file structured inputs, data restructuring. It is HIGH.
		\item Device dependent Operations: I/O processing includes device selection, status checking and error processing. It is NOMINAL.
		\item Computational operations: Use of standard math, basic matrix and vector operations. It is NOMINAL.
		\item Control operations: Reentrant and recursive coding. Task synchronization, complex callbacks. It is VERY HIGH.
	\end{itemize}
	Due to this analysis, the CPLX parameter is \textbf{NOMINAL}.
	\item \textbf{Documentation Match to Life Cycle Needs - DOCU} This cost driver is evaluated in terms of the suitability
of the project's documentation. Since the documentation of the project is supposed to be right-sized to life cycle needs, this parameter is \textbf{NOMINAL}
\end{itemize}

\begin{longtable}{ccc}
\multicolumn{3}{c}{\textbf{RCPX}}\\
\toprule
Cost Driver&Value&Numerical Value\\
\midrule
RELY&High&4\\
DATA&&\\
CPLX&Nominal&3\\
DOCU&Nominal&3\\
\midrule
\multicolumn{2}{c}{\textbf{Total}}&\textbf{BOH}\\
\bottomrule
\end{longtable}

Therefore PERS is \textbf{BOH}.\todo{complete}

\paragraph{Developed for Reusability - RUSE} This cost driver accounts for the additional effort needed to construct components intended for reuse on current or future projects. This effort is consumed with creating more generic design of software, more elaborate documentation, and more extensive testing to ensure components are ready for use in other applications.

Since some project components must cover general aspects of different possible software projects, this parameter is \textbf{NOMINAL}.

\paragraph{Platform Difficulty - PDIF}
\begin{itemize}
	\item \textbf{Execution Time Constraint - TIME} This is a measure of the execution time constraint imposed upon a software system. The rating is expressed in terms of the percentage of available execution time expected to be used by the system or subsystem consuming the execution time resource. Since the system may have exceptional spikes of usage, the idle resource allocation should not exceed 70\% of available execution time. Therefore this parameter is \textbf{HIGH}.
	\item \textbf{Main Storage Constraint - STOR} This rating represents the degree of main storage constraint imposed on a software system or subsystem. Since the system will not require an incremental disk space except for the database whose size is trivial w.r.t modern disk capacity, this parameter is \textbf{NOMINAL}.
	\item \textbf{Platform Volatility - PVOL} "Platform" is used here to mean the complex of hardware and software (OS, DBMS, etc.) the software product calls on to perform its tasks. Our system will use a web server, a DBMS and some external APIs, we expect major releases of them every 6 months and minor releases every 2 weeks, therefore this parameter is \textbf{NOMINAL}.
\end{itemize}

\begin{longtable}{ccc}
\multicolumn{3}{c}{\textbf{PDIF}}\\
\toprule
Cost Driver&Value&Numerical Value\\
\midrule
TIME&High&4\\
STOR&Nominal&3\\
PVOL&Nominal&3\\
\midrule
\multicolumn{2}{c}{\textbf{Total}}&\textbf{10}\\
\bottomrule
\end{longtable}

Therefore PDIF is \textbf{NOMINAL}.

\paragraph{Personnel Experience - PREX}
\begin{itemize}
	\item \textbf{Applications Experience - APEX} The rating for this cost driver is dependent on the level of applications experience of the project team developing the software system. Since all members of the team have an equivalent level of experience with this type of application which is approximately 4 months, this parameter is \textbf{VERY LOW}.
	
	\item \textbf{Platform Experience - PLEX} This parameter reflect the ability to understand and use powerful platforms, including more graphic user interface, database, networking and so on. Since all members of the team have already deal (in different teams) with projects that relied on platforms similar to the ones that will be used for the PowerEnJoy project for an average period of 2 years, this parameter is set to \textbf{NOMINAL}.
	
	\item \textbf{Language and Tool Experience - LTEX} This is a measure of the level of programming language and software tool experience of the project team developing the software system or subsystem. All team members have quite good knowledge of many development best practices and software tools that will be used such as version control systems, integrated development environments, testing frameworks, development quality control software, etc. \todo{language used} This parameter is \textbf{HIGH}.
	
\end{itemize}

\begin{longtable}{ccc}
\multicolumn{3}{c}{\textbf{PREX}}\\
\toprule
Cost Driver&Value&Numerical Value\\
\midrule
APEX&Very Low&1\\
PLEX&Nominal&3\\
LTEX&High&4\\
\midrule
\multicolumn{2}{c}{\textbf{Total}}&\textbf{8}\\
\bottomrule
\end{longtable}

Therefore PREX is \textbf{LOW}.

\paragraph{Facilities - FCIL}
\begin{itemize}
	\item \textbf{Use of Software Tool - TOOL} Since strong, mature lifecycle tools, moderately integrated\todo{decide}
	\item \textbf{Multisite Development - SITE} Since all team members and stakeholders live in the same metropolitan area, this parameter is \textbf{HIGH}.
\end{itemize}

\begin{longtable}{ccc}
\multicolumn{3}{c}{\textbf{FCIL}}\\
\toprule
Cost Driver&Value&Numerical Value\\
\midrule
TOOL&&1\\
SITE&High&4\\
\midrule
\multicolumn{2}{c}{\textbf{Total}}&\textbf{}\\
\bottomrule
\end{longtable}

Therefore FCIL is \textbf{BOH}.\todo{decide}

\paragraph{Required Development Schedule - SCED}
This rating measures the schedule constraint imposed on the project team developing the
software.

\paragraph{Cost Drivers summary}
In table is reported the value of each cost driver as determined in this section and the corresponding numerical valued as specified in \autoref{tbl:costDriverConversionValue}

\begin{longtable}{ccc}
Cost Driver & Value & Numerical Value\\
\midrule
PERS&Nominal&1\\
RCPX&BOH&\\
RUSE&Nominal&1\\
PDIF&Nominal&1\\
PREX&Low&1.22\\
FCIL&BOH\\
SCED&BOH\\
\bottomrule
\caption{\label{tbl:CDChosenValues}Cost drivers chosen values}
\end{longtable}


\subsubsection{Effort equation}

\subsubsection{Schedule estimation } 