\section{Risk management}\label{sec:riskManagement}
This section describe possible risks for the \emph{PowerEnJoy system} and some possible reactive and proactive strategies to mitigate them.
\subsection{Project risks}
\begin{longtable}{lp{0.8\linewidth}}
\multicolumn{2}{c}{\textbf{Failure to achieve a deadline}}\\
\toprule
\textbf{Description}& Various contingencies could bring to fail to achieve deadlines. \\
\midrule
\textbf{Probability}&Moderate\\
\midrule
\textbf{Effects}&Moderate\\
\midrule
\textbf{Actions}& Planning the project schedule it is quite impossible to consider all possible issues that may arise during the project development, so it could be useful to allocate some extra time after each major phase of the project in order to avoid a tasks shift on the schedule due to a fail to achieve a deadline.\\
\bottomrule
\end{longtable}

\begin{longtable}{lp{0.8\linewidth}}
\multicolumn{2}{c}{\textbf{Change of requirements}}\\
\toprule
\textbf{Description}& Requirements may need some changes for different causes: not exhaustive requirements, misunderstood requirements or missing requirements. \\
\midrule
\textbf{Probability}&Moderate\\
\midrule
\textbf{Effects}&Moderate/Serious\\
\midrule
\textbf{Actions}& To avoid the need to change requirements it is really useful to meet all the project stakeholders from the beginning of the development of the project (for example also allowing some users to interact with mockups of the application during the requirements design phase). 

The change of requirements during the development phase could bring to the need to change already implemented software and/or software design. The relevance on the project of that change must be evaluted. If the issue is serious and is customer fault it could be necessary to re-discuss project contract, schedule and estimated budget. 

Anyway a modular and reusable thinking of the software architecture could help to be more reactive to the need of change some requirements.\\
\bottomrule
\end{longtable}

\begin{longtable}{lp{0.8\linewidth}}
\multicolumn{2}{c}{\textbf{Personnel shortfall}}\\
\toprule
\textbf{Description}& Some member of the development team may leave the project; since the size of the team is quite small it may cause serious effects. \\
\midrule
\textbf{Probability}&Low\\
\midrule
\textbf{Effects}&Serious\\
\midrule
\textbf{Actions}& In order to prevent serious effect multiple the project must be well documented in each phase, meetings must be organized to allow each member of the group to have a general knowledge of each part of the project and critic tasks must not be assigned to a single team member.\\
\bottomrule
\end{longtable}


\subsection{Technical risks}
\begin{longtable}{lp{0.8\linewidth}}
\multicolumn{2}{c}{\textbf{Underestimation of requests' load}}\\
\toprule
\textbf{Description}&Traffic and requests are much more than what has been estimated, resulting in high latency and possibly in service unavailability\\
\midrule
\textbf{Probability}&Low\\
\midrule
\textbf{Effects}&Serious\\
\midrule
\textbf{Actions}&Analyze the traffic through all component, consider an upgrade of hardware components, external API contracts and traffic plans. During the development use a modular software design to manage multiple requests in parallel and to improve scalability.\\
\bottomrule
\end{longtable}

\begin{longtable}{lp{0.8\linewidth}}
\multicolumn{2}{c}{\textbf{Data Loss}}\\
\toprule
\textbf{Description}&Data of the system may be compromised due to an hardware fault or to wrong writing operations.\\
\midrule
\textbf{Probability}&Low\\
\midrule
\textbf{Effects}&Serious\\
\midrule
\textbf{Actions}&It is convenient to store an updated backup copy of data possibly on a different location from the system database one.\\
\bottomrule
\end{longtable}

\begin{longtable}{lp{0.8\linewidth}}
\multicolumn{2}{c}{\textbf{Wrong external component specification}}\\
\toprule
\textbf{Description}&An external component is not compliant to its specification\\
\midrule
\textbf{Probability}&Low\\
\midrule
\textbf{Effects}&Serious\\
\midrule
\textbf{Actions}&Try to find a solution with the component's owner, if impossible find a different compatible component and consider to take legal proceedings.\\
\bottomrule
\end{longtable}

\begin{longtable}{lp{0.8\linewidth}}
\multicolumn{2}{c}{\textbf{Difficulties during development phase}}\\
\toprule
\textbf{Description}& It could happen developers experience difficulties in the development of some specific features of the system due to the lack of experience. \\
\midrule
\textbf{Probability}&Low\\
\midrule
\textbf{Effects}&Moderate\\
\midrule
\textbf{Actions}& It may be required to think about asking consulting to external companies.\\
\bottomrule
\end{longtable}

\begin{longtable}{lp{0.8\linewidth}}
\multicolumn{2}{c}{\textbf{Car supplier changes}}\\
\toprule
\textbf{Description}& It could happen the company for some reason needs to change the car supplier. Since our system relies on the car embedded system provided from our supplier it would be a considerable risk if the system must change the communication protocol used with cars. \\
\midrule
\textbf{Probability}&Low\\
\midrule
\textbf{Effects}&Serious\\
\midrule
\textbf{Actions}& To mitigate the possible effects of this situation the system software must abstract from the type of communication protocol used with cars and interface components must be designed to be as reusable as possible in order to reduce to minimum needed changes.\\
\bottomrule
\end{longtable}

\begin{longtable}{lp{0.8\linewidth}}
\multicolumn{2}{c}{\textbf{APIs bad behaviour}}\\
\toprule
\textbf{Description}& The system needs to interact with some external APIs that can not behave as expected. \\
\midrule
\textbf{Probability}&Low\\
\midrule
\textbf{Effects}&Moderate\\
\midrule
\textbf{Actions}& External APIs interaction must be deeply tested during the development phase in order to identify possibly unexpected behaviours in this phase: company owner of the API service could be contacted to eventually ask to fix the behaviour or, if it is related to an open source project, a pull request may be proposed to fix the behaviour. \\
\bottomrule
\end{longtable}



\subsection{Business risks}
\begin{longtable}{lp{0.8\linewidth}}
\multicolumn{2}{c}{\textbf{Competitors}}\\
\toprule
\textbf{Description}&Competitors saturate the market with their car sharing services\\
\midrule
\textbf{Probability}&High\\
\midrule
\textbf{Effects}&Serious\\
\midrule
\textbf{Actions}&Make some market surveys in order to determine and implement some \emph{killer features} to persuade users to use the \emph{PowerEnJoy} service.\\
\bottomrule
\end{longtable}

\begin{longtable}{lp{0.8\linewidth}}
\multicolumn{2}{c}{\textbf{Financial crisis}}\\
\toprule
\textbf{Description}&Customer may have financial troubles during the development of the system causing a reduction of the budget.\\
\midrule
\textbf{Probability}&Low\\
\midrule
\textbf{Effects}&Catastrophic\\
\midrule
\textbf{Actions}& A good study of feasibility would detect this risk.  This situation could be mitigate meeting the customer and re-discussing features and agreed requirements but could also bring to the end of the project development. \\
\bottomrule
\end{longtable}

\begin{itemize}

\item Car sharing non sfonda -> difficoltà fondi per mantenimento software
\item Securityl
\item Unrealistic schedule/budget
\item Wrong functionality
\item Wrong user interface (not usable)
\item Goldplating -> Perfezionismo
\item Requirements volatility
\item Bad external components -> soprattutto difficulty interface with car
\item Bad external tasks
\item Real-time shortfalls
\item Capability shortfalls
\item API downtime
\item API changes
\end{itemize}

