\section{Introduction}
\subsection{Purpose of this document}
The purpose of a \textbf{R}equirement \textbf{A}nalysis and \textbf{S}pecifications \textbf{D}ocument is the process of discovering the purpose for which a software system was intended, by identifying stakeholders and their needs, and documenting these in a form that is amenable to analysis, communication, and subsequent implementation. \todo{Citation} It is also concerned with the relationship of software's factors such as goals, functions and constrains to precise specifications of software behavior, and to their evolution over time and across software families.

\subsection{Actual system}
The system we are to develop is brand new, there is no previous system.

\subsection{Scope}
PowerEnJoy is a car-sharing service that exclusively employs electric cars; we are going to develop a web-based software system that will provide the functionalities normally provided by car-sharing services, such as allowing the user to register to the system in order to access it, showing the cars available near a given location and allowing a user to reserve a car before picking it up.
A screen located inside the car will show in real time the ride amount of money to the user. When the user reaches a predefined safe area and exits the car, the system will stop charging the user and will lock the car. The system will provide information about charging station location where the car can be plugged after the ride and incentivize virtuous behaviours of the users with discounts.
	\subsubsection{Goals}
	\begin{description}
		\item[G1] Allow users to register to the system
		\item[G2] Allow \todo{aggiungere glossary}registered users to authenticate to the system
		\item[G3] Provide authenticated users with the position of \todo{definire available/usable [available][reserved][in use]$\triangle$[not available][under maintenance]}available cars
		\item[G4] Notify PowerEnJoy maintenance operators with a list of not available cars \todo{glossary di available, parola giusta?}
		\item[G5] Provide PowerEnJoy maintenance operators with an interface to handle not available cars, under a special maintenance condition \todo{da definire nel glossario la maintenance condition}
		\item[G6] Provide PowerEnJoy customer service with an interface to ban\todo{glossary} users in order to prevent them from reserving or using other cars, and enable them to use the service again
		\item[G7] Ban a user when the payment transaction for its last ride was unsuccessful and notify PowerEnJoy customer service of the missing payment
		\item[G8] Allow a user to reserve a car, if available, and hold this reservation for an hour
		\item[G9] In case after and hour, a reserved car wasn't used the user who reserved it will be charged 1\euro
	\end{description}

\subsection{Glossary}
	\subsubsection{Definitions}
	\begin{description}
		\item[System:]the PowerEnJoy software we are to develop
		\item[Company:] the company we are developing the system for
		\item[Guest \emph{or} Guest user:] person who access the system as non logged user
		\item[Logged user \emph{or} Authenticated user:] authenticated person who is interfacing with the system
		\item[Car:] an electric car owned by the company
		\item[Charging station:] place where cars owned by the company can be charged using a provided plug
		\item[Authentication \emph{or} Login:] interaction between guests and the system that grant authenticated user's privileges to a guest user
		\item[Registration:]
		\item[Safe area:]\todo{What is the safe area?}
	\end{description}
\subsubsection{Acronyms}
	\begin{description}
		\item [RASD:] Requirements Analysis and Specification Document
		\item [API:] Application Programming Interface
		\item [GPS:] Global Position System
	\end{description}
\subsubsection{Abbreviations}
	\begin{description}
		\item [Km:] Kilometers
		\item [w.r.t.:] with respect to
		\item [i.d.:] id est
	\end{description}