\section{Introduction}

\subsection{Purpose of this document}

\subsection{Scope}
PowerEnJoy is a car-sharing service that exclusively employs electric cars; we are going to develop a web-based software system that will provide the functionalities normally provided by car-sharing services, such as allowing the user to register to the system in order to access it, showing the cars available near a given location and allowing a user to reserve a car before picking it up.
A screen located inside the car will show in real time the ride amount of money to the user. When the user reaches a predefined safe area and exits the car, the system will stop charging the user and will lock the car. The system will provide information about charging station location where the car can be plugged after the ride and incentivize virtuous behaviours of the users with discounts \cite{RASD}.

\subsection{Glossary}
The \emph{PowerEnJoy: Requirements Analysis and Specification Document} \cite{RASD}, the \emph{PowerEnJoy: Design Document} \cite{DD} and the \emph{PowerEnJoy: Integration Test Plan Document} \cite{ITPD} should be referenced for terms not defined in this section.

\subsubsection{Acronyms}
	\begin{description}
		\item [RASD:] Requirements Analysis and Specification Document
		\item [DD:] Design Document
		\item [ITPD:] Integration Test Plan Document
		\item [API:] Application Programming Interface
		\item [GPS:] Global Position System
		\item [DB:] DataBase
		\item [DBMS:] DataBase Management System
		\item [GIS:] Geographic Information System
	\end{description}
\subsubsection{Abbreviations}
	\begin{description}
		\item [w.r.t.:] with respect to
		\item [i.d.:] id est
		\item [i.f.f.:] if and only if
		\item [e.g.:] exempli gratia
		\item [etc.:] et cetera
	\end{description}

\subsection{Reference documents}
Context, domain assumptions, goals, requirements and system interfaces are all described in the \emph{PowerEnJoy: Requirements Analysis and Specification Document} \cite{RASD}.\\
Software design and architecture of the system are all described in the \emph{PowerEnJoy: Design Document} \cite{DD}.\\

\subsection{Document overview}
This document is structured as
\begin{enumerate}
	\item \ldots
\end{enumerate}