\section{Individual Steps and Test Description}

For each step of the integration process above, describe the type of tests that will be used to verify that the elements integrated in this step perform as expected. Describe in general the expected results of the test set. You may refer to Chapter 3 and Chapter 4 of the test plan example [1] as an example of what we expect.
(NOTE: This is not a detailed description of test protocols. Think of this as the test design phase. Specific protocols will be written to fulfill the goals of the tests in this section.)

\subsection{Test Description}

\begin{longtable}{p{0.3\linewidth}p{0.7\linewidth}}
\hline \textbf{ExampleFunction(argument1,argument2)} \\
\toprule
\emph{Input} & \emph{Expected output} \\
\midrule
A null parameter & A NullArgumentException is raised.\\
\midrule
A non valid argument1 & An InvalidArgumentValueException is
raised. \\
\midrule
A set of valid parameters & The example function is executed and 42 is returned. \\
\bottomrule
\caption{\emph{ExampleFunction} Test description}
\end{longtable}

\subsubsection{RentManager, DataProvider}

%getFromFile(searchFor)
\begin{longtable}{p{0.4\linewidth}p{0.6\linewidth}}
\multicolumn{2}{c}{\textbf{getFromFile(searchFor)}} \\
\toprule
\emph{Input} & \emph{Expected output} \\
\midrule
A null searchFor & A NullArgumentException is raised.\\
\midrule
The corresponding file does not exists & A FileNotFoundException is raised. \\
\midrule
The parameter value does not correspond to any known file & An InvalidArgumentValueException is raised. \\
\midrule
The parameter value corresponds to a known existing file & The pointer to the correct file is returned. \\
\bottomrule
\caption{\emph{getFromFile(searchFor)} test description}
\end{longtable}


%createCarJPA(carID)
\begin{longtable}{p{0.4\linewidth}p{0.6\linewidth}}
\multicolumn{2}{c}{\textbf{createCarJPA(carID)}} \\
\toprule
\emph{Input} & \emph{Expected output} \\
\midrule
A null carID & A NullArgumentException is raised.\\
\midrule
A carID which does not correspond with any car in the DB & An InvalidArgumentValueException is raised. \\
\midrule
A carID which identifies a car in the DB & A JPA class mapped to the car identified by the carID in the DB is returned. \\
\bottomrule
\caption{\emph{createCarJPA(carID)} test description}
\end{longtable}


%createReservationJPA(carID, userID)
\begin{longtable}{p{0.4\linewidth}p{0.6\linewidth}}
\multicolumn{2}{c}{\textbf{createReservationJPA(carID, userID)}} \\
\toprule
\emph{Input} & \emph{Expected output} \\
\midrule
A null carID and/or a null userID & A NullArgumentException is raised.\\
\midrule
A carID which does not correspond to any car and/or a userID does not correspond to any user in the DB & An InvalidArgumentValueException is raised. \\
\midrule
The user identified by the userID has already more than zero reservation & An AtMostOneReservationException is raised. \\
\midrule
The user identified by the userID is performing a rent & An NoReservationDuringARentException is raised. \\
\midrule
The carStatus DB attribute of the car identified by the carID is not \emph{Available} & A NotReservableCarException is raised. \\
\midrule
The carStatus DB attribute of the car identified by the carID is \emph{Available} and the user identified by the userID has no active reservations or rents & \begin{itemize}
	\item A new reservation for the specified car and user and with the current timestamp is created in the DB
	\item the carStatus attribute of the specified car changes to \emph{Reserved}
	\item the JPA class mapped to the new reservation is returned.
\end{itemize} \\
\bottomrule
\caption{\emph{createReservationJPA(carID, userID)} test description}
\end{longtable}

%createPaymentJPA(rent, discounts, fees, paymentStatus)
\begin{longtable}{p{0.3\linewidth}p{0.7\linewidth}}
\multicolumn{2}{c}{\textbf{createPaymentJPA(rent, discounts, fees, paymentStatus)}} \\
\toprule
\emph{Input} & \emph{Expected output} \\
\midrule
A null rent and/or a null discounts array and/or a null fees array and/or a null paymentStatus & A NullArgumentException is raised.\\
\midrule
A rent with endTimestamp and/or endLocation null attributes & An InvalidArgumentValueException is raised. \\
\midrule
An unknown paymentStatus & An InvalidArgumentValueException is raised. \\
\midrule
\begin{itemize}
\item An empty array of discounts and/or an empty array of fees
\item A not empty array of discounts and/or a not empty array of fees
\end{itemize} & The applied fee and applied discount are computed, a new payment is inserted properly in the DB with the paymentStatus attribute as \emph{Pending}\\
\bottomrule
\caption{\emph{createPaymentJPA(rent, discounts, fees, paymentStatus)} test description}
\end{longtable}

%createRentJPA(startLocation, startTimestamp, car, msoStation)
\begin{longtable}{p{0.3\linewidth}p{0.7\linewidth}}
\multicolumn{2}{c}{\textbf{createRentJPA(startLocation, startTimestamp, car, msoStation)}} \\
\toprule
\emph{Input} & \emph{Expected output} \\
\midrule
A null startLocation and/or a null startTimer and/or a null car & A NullArgumentException is raised.\\
\midrule
A startTimestamp in the future & An InvalidArgumentValueException is raised. \\
\midrule
A wrong format of startTimestamp & An InvalidArgumentValueException is raised. \\
\midrule
A wrong format of startLocation & An InvalidArgumentValueException is raised. \\
\midrule
The carStatus DB attribute of the specified car is not \emph{Reserved} & A CarNotReservedException is raised. \\
\midrule
A msoStation parameter which does not identify any charging station in the DB & An InvalidArgumentValueException is raised. \\
\midrule
Proper startLocation, startTimestamp and car parameters, null msoStation & A new payment (paired with the user that has reserved the car) is inserted properly in the DB with endTimestamp, endLocation and moneySavingOption attributes set to null.\\
\midrule
Proper \mbox{startLocation}, startTimestamp and car parameters, \mbox{msoStation} parameter valid and not null & A new payment (paired with the user that has reserved the car) is inserted properly in the DB with endTimestamp, endLocation attributes set to null.\\
\bottomrule
\caption{\emph{createRentJPA(startLocation, startTimestamp, car, msoStation)} test description}
\end{longtable}

%findReservedCar(userID)
\begin{longtable}{p{0.3\linewidth}p{0.7\linewidth}}
\multicolumn{2}{c}{\textbf{findReservedCar(userID)}} \\
\toprule
\emph{Input} & \emph{Expected output} \\
\midrule
A null userID & A NullArgumentException is raised.\\
\midrule
An unknown userID  & An InvalidArgumentValueException is raised. \\
\midrule
A valid userID identifying a user without an active reservation & A NoReservedCarException is raised. \\
\midrule
A valid userID identifying a user with an active reservation & A CarJPA related to the car currently reserved by the user associated with the userID argument is returned.\\
\bottomrule
\caption{\emph{findReservedCar(userID)} test description}
\end{longtable}

\subsubsection{RentManager, CarHandler}
getBattery(listOfCars) \\
unlock(carJPA) \\
trigger(carJPA, myTriggers) \\
removeTrigger(carJPA, myTriggers) \\
getParameters(parameters) \\ %parameters example such as batteryLevel, passengers ecc...

\subsubsection{MaintenanceManager, DataProvider}
getFailuresToBeFixed()\\
fixFailureTagAvailable(failureID)

\subsubsection{MaintenanceManager, CarHandler}
getSoftwareKeys(carsList) 

\subsubsection{UserAppServer, RentManager}
getMapAvailableCars(position) \\
reserveCar(userID, carID, MSOOptionBoolean) \\
carUnlock(userID) \\

\subsubsection{CustomerCareServer, DataProvider}
setUserStatus(userID, bannedBoolean) \\
tagCarAsNotAvailable(carID) \\


\paragraph{UserAppServer, AccessManager}
login(username, password) %returns user ID

\paragraph{AccessManager, DataProvider} 
description \\
query(queryToMake)

\paragraph{UserInformationManager, DataProvider} 
description \\
query(queryToMake)

\paragraph{UserAppServer, UserInformationManager}
getUserInfoUAS(userID)
getRentHistory(userID) \\
getPaymentHistory(userID) \\

\paragraph{CustomerCareServer, UserInformationManager}
getUserInfoCCS(userID) \\
getUsers()
getRentHistory(userID) \\
getPaymentHistory(userID) \\










