\section{Introduction}

\subsection{Purpose of this document}
The purpose of the Power EnJoy Project Plan is to provide an overview of the project with respect of parameters related to the project management. This document provides an estimated expected cost and effort for the project based upon techniques to support the estimation
procedure (Function Points, COCOMO II), proposes a possible schedule for the project and a related allocation of resources to accomplish tasks identified, describes the possible risks related to the project presenting some possible plans to manage them.


\subsection{Scope}
PowerEnJoy is a car-sharing service that exclusively employs electric cars; we are going to develop a web-based software system that will provide the functionalities normally provided by car-sharing services, such as allowing the user to register to the system in order to access it, showing the cars available near a given location and allowing a user to reserve a car before picking it up.
A screen located inside the car will show in real time the ride amount of money to the user. When the user reaches a predefined safe area and exits the car, the system will stop charging the user and will lock the car. The system will provide information about charging station location where the car can be plugged after the ride and incentivize virtuous behaviours of the users with discounts \cite{RASD}.

\subsection{Glossary}
The \emph{PowerEnJoy: Requirements Analysis and Specification Document} \cite{RASD}, the \emph{PowerEnJoy: Design Document} \cite{DD} and the \emph{PowerEnJoy: Integration Test Plan Document} \cite{ITPD} should be referenced for terms not defined in this section.

\subsubsection{Acronyms}
	\begin{description}
		\item [RASD:] Requirements Analysis and Specification Document
		\item [DD:] Design Document
		\item [ITPD:] Integration Test Plan Document
		\item [FP:] Function Points
		\item [ILF:] Internal logic file
		\item [ELF:] External logic file
		\item [EI:] External Input
		\item [EO:] External Output.
		\item [EQ:] External Inquiries
		\item [COCOMO:] COnstructive COst MOdel
		\item [API:] Application Programming Interface
		\item [GPS:] Global Position System
		\item [DB:] DataBase
		\item [DBMS:] DataBase Management System
		\item [GIS:] Geographic Information System
	\end{description}
\subsubsection{Abbreviations}
	\begin{description}
		\item [w.r.t.:] with respect to
		\item [i.d.:] id est
		\item [i.f.f.:] if and only if
		\item [e.g.:] exempli gratia
		\item [etc.:] et cetera
	\end{description}

\subsection{Reference documents}
\begin{itemize}
	\item Context, domain assumptions, goals, requirements and system interfaces are all described in the \emph{PowerEnJoy: Requirements Analysis and Specification Document} \cite{RASD}
	\item Software design and architecture of the system are all described in the \emph{PowerEnJoy: Design Document} \cite{DD}
	\item System integration plan strategy is described in the \emph{PowerEnJoy: Integration Test Plan Document} \cite{ITPD}
	\item Function Point Counting Practices Manual \\ 
Release 4.3.1 International Function Point Users Group (IFPUG)
	\item Function Point Languages Table \\ 
\url{http://www.qsm.com/resources/function-point-languages-table}
	\item COCOMO II – Model Definition Manual \\
Version 2.1 Center for Software Engineering, USC

\end{itemize}


\subsection{Document overview}
This document is structured as
\begin{enumerate}
	\item \textbf{Introduction}: contains references, glossary, definitions, acronyms and abbreviations; it also explains the purpose and scope of this document
	\item \textbf{Project size, cost and effort estimation}: provides estimations of the expected size, cost and required effort of the \emph{PowerEnJoy system}
	\begin{itemize}
		\item Size estimation: \emph{Function Points} will be used to estimate the size of the \emph{PowerEnJoy system} starting from the complexity of its main functionalities
		\item Cost and effort estimation: the \emph{Constructive Cost Model (COCOMO) II} will be used to estimate the cost and effort needed to develop the \emph{PowerEnJoy system}
	\end{itemize}
	\item \textbf{Schedule}: contains a general high level schedule of the project's activities
	\item \textbf{Resource allocation}: defines how the activities in the schedule will be allocated between the members of the group
	\item \textbf{Risk management}: contains the strategies to manage the main risks that the project development may face
\end{enumerate}