\begin{appendices}

	\section{Software and tools used}
	For the development of this document we used
	\begin{itemize}
		\item \LaTeX{} as document preparation system
		\item Git \& \href{http://github.com}{GitHub} as version control system
		\item \href{http://draw.io}{Draw.io} for graphs 
	\end{itemize}
	
	\section{Hours of work}
	This is the amount of time spent to redact this document:
	\begin{itemize}
		\item Davide Piantella: $\sim$ 
		\item Mario Scrocca: $\sim$ 
		\item Moreno R. Vendra: $\sim$ 
	\end{itemize}
	
\end{appendices}


\begin{thebibliography}{9}

\bibitem{RASD}D. Piantella, M. Scrocca, M.R. Vendra, \emph{PowerEnJoy: Requirements Analysis and Specification Document}, Politecnico di Milano - Software Engineering II Project, 2016

\bibitem{DD}D. Piantella, M. Scrocca, M.R. Vendra, \emph{PowerEnJoy: Design Document}, Politecnico di Milano - Software Engineering II Project, 2016

\bibitem{ITPD}D. Piantella, M. Scrocca, M.R. Vendra, \emph{PowerEnJoy: Integration Test Plan Document}, Politecnico di Milano - Software Engineering II Project, 2017

\bibitem{QSM}Quantitative Software Management, Function Point Languages Table version 5.0, QSM SLOC/FP Data

\bibitem{FP}Function Point Counting Practices Manual \\ 
Release 4.3.1 International Function Point Users Group (IFPUG)

\end{thebibliography}